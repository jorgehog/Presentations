\documentclass{beamer}

\usepackage[utf8]{inputenc}
\usepackage{default, verbatim, graphicx}
\usetheme{Warsaw}
\title[Regex - for the people!]{Introduction to Regex\\Learn to recognize when you should use it.}
\author{Jørgen Høgberget}
\date{}

\begin{document}
\begin{frame}
\titlepage
\end{frame}

\AtBeginSubsection[]
{
  \begin{frame}<beamer>
    \frametitle{Today's show}
    \tableofcontents[currentsection,currentsubsection]
  \end{frame}
}

\section{Regular expressions. What the..?}

\begin{frame}[fragile]
\scriptsize
\begin{itemize}
 \item \verb@anyNumber = r"^?\s*(\d+\.?\d*[eE]?[\-\+]?\d*)\s*\b?"@
\pause \item Above: Regex pattern for any number on the form $1$, $123.456$, $123.456e-123$, $123.456e+123$, etc.
\end{itemize}
\end{frame}

\begin{frame}[fragile]
\scriptsize
\begin{itemize}
 \item \verb@anyNumber = r"\s*(\d+\.?\d*[eE]?[\-\+]?\d*)\s*"@
\pause \item Above: Regex pattern for any number on the form $1$, $123.456$, $123.456e-123$, $123.456e+123$, etc.
\end{itemize}
\end{frame}

\begin{frame}[fragile]
\scriptsize
\begin{verbatim}
>>> import re
>>> text = "asd1 foo 2.0 0.123213 foo foo 1E-2 foo 1e-04 1.2313e123" 
>>> anyNumber = "\s*(\d+\.?\d*[eE]?[\-\+]?\d*)\s*"
>>> re.findall(anyNumber, text)
['1', '2.0', '0.123213', '1E-2', '1e-04', '1.2313e123']
\end{verbatim}
\normalsize
\end{frame}

\begin{frame}[fragile]
\frametitle{Cracking the code}
\scriptsize
\begin{itemize}
 \item Regular expressions takes as input a \textit{pattern} and the \textit{source string}.
 \pause \item \verb+anyNumber+ is a pattern reprisenting any number.
 \item \verb@anyNumber = r"\s*(\d+\.?\d*[eE]?[\-\+]?\d*)\s*"@
 \pause \item \verb@r"string"@ is a \textit{raw string}. These strings does not read special string identifiers such as \verb+\n+ (which might mess up the regexp).
\end{itemize}
\normalsize
\end{frame}

\begin{frame}[fragile]
\frametitle{Cracking the code}
\scriptsize
\begin{itemize}
 \item \verb@anyNumber = r"\s*(\d+\.?\d*[eE]?[\-\+]?\d*)\s*"@
 \item \verb@? + * . ) (@ has special functions. Let's focus on them later.

\end{itemize}
\normalsize
\end{frame}

\begin{frame}[fragile]
\frametitle{Cracking the code}
\scriptsize
\begin{itemize}
 \item \verb@anyNumber = r"\s\d\.\d[eE][\-\+]\d)\s"@
 \pause \item \verb+\s+ - matches any whitespace character (tabs, newlines, etc.)
 \pause \item \verb+\d+ - matches any integer
 \pause \item \verb+\.+ - matches a comma (we have to use \verb+\+ since . has a special function. (like \% in TeX))
 \pause \item \verb+[eE]+ - matches either e or E 
 \item \verb@[\+\-]@ - matches either + or - (again backslash since + and - has special function.)
 \pause \item Summarized: Any number is isolated by spaces, starts with an integer, then a comma, then an integer, then e or E, then + or -, then an integer.
 \item This is obviously not true, but in the \textit{right combinations} it is true. Let's go back to the special characters.
\end{itemize}
\normalsize
\end{frame}

\begin{frame}[fragile]
\frametitle{Cracking the code}
\scriptsize
\begin{itemize}
 \item \verb@anyNumber = r"\s*(\d+\.?\d*[eE]?[\-\+]?\d*)\s*"@
 \pause \item \verb@anyNumber =     *(  +  ?  *    ?      ?  *)  * @  
 \item These special characters describe allowed combinations of the previous regexp.
 \pause \item \verb+?+ : Optional. It is either there or not. In other words: The comma is optional, e or E is optional, + or - is optional.
 \pause \item \verb@+@ : Not Optional. Allows for multiple successing regexp of the previous statement. \verb@\d+@ matches any integer.
 \pause \item \verb+*+ : Optional combined with allowing any amount of the previous regex to success eachother.
 \pause \item \verb@(...)@ : Identifies a \textit{group}. When we apply \verb+re.findall+, we want the numbers returned, not the spaces (\verb+\s*+) (even though they are a part of the regexp pattern); we put the matching number in a group.
\end{itemize}
\normalsize
\end{frame}


\begin{frame}[fragile]
\frametitle{Cracking the code}
\scriptsize
\begin{itemize}
\item \verb@anyNumber = r"\s*(\d+\.?\d*[eE]?[\-\+]?\d*)\s*"@
\pause \item Does this expression allow for negative numbers?
\pause \item Frank, Svenn-Arne; you are not allowed to answer.
\pause \item Anyone else?
\pause \item You are right! It does not!
\item So what do we need to add?
\pause \item A precurring \verb+-?+ (NOTE: Inside the group!)
\pause \item We do not need a backslash since - is only special inside brackets. (\verb+[a-z]+ means any character between a and z will match)
\end{itemize}
\normalsize
\end{frame}

\begin{frame}[fragile]
\frametitle{Cracking the code}
\scriptsize
\begin{itemize}
\item \verb@anyNumber = r"\s*(-?\d+\.?\d*[eE]?[\-\+]?\d*)\s*"@
\item Does this expression allow for negative numbers?
\item Frank, Svenn-Arne; you are not allowed to answer.
\item Anyone else?
\item You are right! It does not!
\item What do we need to add?
\item A precurring \verb+-?+
\item We do not need a backslash since - is only special inside brackets. (\verb+[a-z]+ means any character between a and z will match)
\end{itemize}
\normalsize
\end{frame}

\section{Examples}
\subsection{Retrieving data from raw output}

\begin{frame}[containsverbatim]
\scriptsize
\begin{verbatim}
...
dmcE: 2.99999| Nw:  998| 92.30000%
dmcE: 3.00000| Nw:  995| 92.40000%
dmcE: 3.00000| Nw:  995| 92.50000%
dmcE: 3.00000| Nw:  996| 92.60000%
...
dmcE: 2.99999| Nw:  998| 94.00000%
dmcE: 3.00000| Nw:  999| 94.10000%
dmcE: 3.00000| Nw:  999| 94.20000%
dmcE: 3.00000| Nw:  998| 94.30000%
dmcE: 3.00000| Nw:  995| 94.40000%
dmcE: 3.00001| Nw:  992| 94.50000%
dmcE: 3.00001| Nw:  993| 94.60000%
...
dmcE: 3.00002| Nw:  984| 99.80000%
dmcE: 3.00001| Nw:  984| 99.90000%
dmcE: 3.00002| Nw:  985| 100.00000%
DMC FIN.
Job fin
\end{verbatim}
\normalsize
\end{frame}

\begin{frame}[containsverbatim]
\scriptsize
\begin{verbatim}
def getDmcE(path):
    stdout = open(path + "/stdout.txt", 'r')
    stdoutRaw = "\n".join(stdout.readlines())
    stdout.close()
    
    pattern = "dmcE:\s*(\d+\.?\d*)\s*\|\s*Nw:\s*\d+\|\s*100\.?[0]*%"
    
    r = re.findall(pattern, stdoutRaw)
    
    if r:
        #r[0] = therm; r[1] = production

        #Both thermalization and main cycles succeeded.
        if len(r) ==2:
               return float(r[1])
        else:
        #Run aborted after thermalization.
               return "~" + r[0]
    else:
        #Run aborted.
        return "N/A"
\end{verbatim}
\normalsize
\end{frame}


\subsection{Analyzing an Austrian's master thesis}

\begin{frame}
\begin{center}
 \includegraphics[scale=0.1]{christoffer_1.jpg}
\end{center}
\end{frame}

\begin{frame}
\begin{center}
 \includegraphics[scale=0.133]{christoffer_2.jpg}
\end{center}
\end{frame}




\begin{frame}[containsverbatim]
\begin{itemize}
 \item Step one: Concatinate all the his tex-files into one file:  
 \vspace{0.2cm}
\scriptsize
\verb+~$ find . -name *.tex -exec cat {} \; > ~/.../christoffer_raw.tex+
\normalsize

 \item Use \verb+count_words.py+ to count the accurances of important strings/words such as ...

\end{itemize}
\end{frame}

\begin{frame}[containsverbatim]
\begin{itemize}

 \item Jørgen, Coupled Cluster and Monte Carlo
 \vspace{0.2cm}
\scriptsize
\begin{verbatim}
~$ python count_words.py -i -b christoffer_raw.tex Jørgen
Number of occurances of word 'Jørgen' (case insensitive): 1

~$ python count_words.py -i christoffer_raw.tex "coupled cluster"
Number of occurances of string 'coupled cluster' (case insensitive): 15

~$ python count_words.py -i christoffer_raw.tex "monte carlo"
Number of occurances of string 'monte carlo' (case insensitive): 2

\end{verbatim}
\normalsize
\end{itemize}
\end{frame}


\begin{frame}[containsverbatim]
\tiny
\begin{verbatim}
...cml-line parsing...

#Adding bounds to the word (pattern=any spacing, word, any ending)
stringOrWord = "string"
if bounded:
    stringOrWord = "word"
    word = "[\^\s]" + word + "[\s\b]"
    
#Flaging case insensitivity
if not caseSense:
    regExtObj = re.compile(word, re.IGNORECASE)
    printCase = " (case insensitive)"
else:
    regExtObj = re.compile(word)
    printCase = ""

Nmatches = len(regExtObj.findall(rawFile))

printfArgs = (stringOrWord, rawWord, printCase, Nmatches)
print "Number of occurances of %s '%s'%s: %d" %  printfArgs
\end{verbatim}
\normalsize
\end{frame}





\begin{frame}[containsverbatim]
\begin{itemize}

\item Use \verb+list_occurances.py+ to reveal his most used words:
 
\vspace{0.2cm}
\scriptsize
\begin{verbatim}
christoffer_raw.tex           
word           : n              

electron       : 498            
tension        : 182            
omega          : 134            
basis          : 130            
particle       : 126            
operators      : 126            
operator       : 117            
delta          : 114            
state          : 102            
electrons      : 92             
states         : 89    
...
\end{verbatim}
\normalsize
\end{itemize}
\end{frame}

\begin{frame}[containsverbatim]
\tiny
\begin{verbatim}
import sys, re

#load entire file into a string. Convert to lower case letters.
f1 = open(sys.argv[1]); allwords = f1.read().lower(); f1.close()

max_length = 15; min_length = 5

#Recognize all words
rePattern = r"[a-zA-Z]{%d,%d}" % (min_length, max_length)
allwords = re.findall(rePattern, allwords)

texWords = ... list of texWords such as begin, end, left, right..

#count words
wordCount = {}
for word in allwords:
    if word not in texWords and not re.findall('.*(fmf).*', word):
        if word not in wordCount.keys():
            wordCount[word] = 1
        elif word in wordCount.keys():
            wordCount[word] += 1

#Sort scores from largest to lowest
sort = sorted(wordCount.items(), key=lambda x: x[1], reverse=True)

#Output:
s = max_length
print sys.argv[1].ljust(len(sys.argv[1]))
print "%s: %s" % ("word".ljust(s), "n".ljust(s))     
print
for key, value in sort:
    print "%s: %s" % (key.ljust(s), str(value).ljust(s))
\end{verbatim}
\normalsize
\end{frame}

\end{document}
